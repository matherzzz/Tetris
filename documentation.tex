\documentclass{article}
\usepackage{enumitem}
\usepackage{hyperref}

\begin{document}

\title{Документация для проекта BrickGame v1.0 aka Tetris}
\date{}
\maketitle

\section{Введение}

Проект BrickGame v1.0 aka Tetris представляет собой реализацию классической аркадной игры Тетрис с использованием языка программирования C и библиотеки ncurses для терминального интерфейса.

\section{Сборка проекта}

Проект использует систему сборки \texttt{make} с Makefile, включающим следующие цели:

\begin{itemize}
    \item \texttt{all}: Сборка проекта.
    \item \texttt{install}: Установка программы.
    \item \texttt{uninstall}: Удаление программы.
    \item \texttt{clean}: Очистка временных файлов и папок.
    \item \texttt{dvi}: Создание файла DVI.
    \item \texttt{dist}: Создание архива, содержащего необходимые файлы для сборки и использования программы.
\end{itemize}

\section{Требования к среде выполнения}

Проект предполагает использование языка программирования C11, компилятора gcc и библиотеки ncurses для терминального интерфейса.

\section{Использование программы}

\begin{enumerate}
    \item \textbf{Управление:}
        \begin{itemize}
            \item Используйте стрелки влево и вправо для перемешения фигуры по горизонтали.
            \item Нажмите клавишу вниз для падения фигуры.
            \item Для поворота фигуры используйте стрелку вверх.
            \item Для постановки игры на паузу используйте пробел.
            \item Для выхода из игры используйте escape.
        \end{itemize}
    \item \textbf{Завершение игры:}
        \begin{itemize}
            \item Игра завершается, когда достигнута верхняя граница игрового поля.
        \end{itemize}
\end{enumerate}